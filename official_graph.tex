\documentclass{article}
\usepackage{esvect}

% Language setting
% Replace `english' with e.g. `spanish' to change the document language
\usepackage[vietnamese]{babel}

% Set page size and margins
% Replace `letterpaper' with `a4paper' for UK/EU standard size
\usepackage[letterpaper,top=2cm,bottom=2cm,left=3cm,right=3cm,marginparwidth=1.75cm]{geometry}
\usepackage{amsmath}
\usepackage{amsfonts}
\usepackage{mathtools}

% Useful packages
\usepackage{amsmath}
\usepackage{graphicx}
\usepackage[colorlinks=true, allcolors=blue]{hyperref}
\usepackage{tcolorbox}
\usepackage{mathtools}
\usepackage{tikz}
\usepackage{tikz-3dplot} % For 3D plotting
\usepackage{pgfplots}
\pgfplotsset{compat=1.17} % Sử dụng phiên bản tương thích mới

\begin{document}

\maketitle

Phần đồ thị:\\
\underline{Đồ thị 1: $U_{hd}$ trong trường xuyên tâm:}
\begin{center}
    \begin{tikzpicture}
        % Định nghĩa các hằng số mẫu (bạn có thể thay đổi)
        \def\C{1} % Giá trị đại diện cho M^2 / (2mr^2)
        \def\E{0.5} % Giá trị năng lượng E
        \pgfmathsetmacro{\rmin}{sqrt(\C/\E)} % Tính r_min từ E = C / r_min^2
    
        \begin{axis}[
            axis lines=middle, % Trục tọa độ đi qua gốc
            xlabel=$r$,          % Nhãn trục hoành
            ylabel={$U_{\text{hd}}$}, % Nhãn trục tung
            xlabel style={anchor=north east}, % Vị trí nhãn trục hoành
            ylabel style={anchor=south east}, % Vị trí nhãn trục tung
            xmin=0,
            xmax=4,          % Giới hạn trục hoành (điều chỉnh nếu cần)
            ymin=0,
            ymax=3,          % Giới hạn trục tung (điều chỉnh nếu cần)
            xtick={\rmin},     % Đặt tick tại r_min
            xticklabels={$r_{\min}$}, % Nhãn cho tick r_min
            ytick={\E},        % Đặt tick tại E
            yticklabels={$E$},   % Nhãn cho tick E
            samples=100,       % Số điểm mẫu để vẽ đường cong mượt hơn
            domain=0.5:4,     % Miền giá trị của r để vẽ (tránh r=0)
            no markers,        % Không hiển thị dấu điểm trên đồ thị
            every axis plot/.append style={thick}, % Làm đậm các đường vẽ
        ]
    
        % Vẽ đường cong U_hd = M^2 / (2mr^2) (sử dụng C = 1)
        \addplot[blue, smooth] { \C / (x^2) }
            node[pos=0.3, anchor=west, font=\small] {$\frac{M^2}{2mr^2}$}; % Thêm nhãn cho đường cong
    
        % Vẽ đường ngang tại E
        \addplot[red, domain=0:4] { \E }; % Vẽ từ r_min trở đi
    
        % Vẽ đường nét đứt từ (r_min, 0) đến (r_min, E)
        \draw[dotted] (axis cs: \rmin, 0) -- (axis cs: \rmin, \E);
    
        \end{axis}
    \end{tikzpicture}
\end{center}
\underline{Đồ thị 2: Hình vẽ quỹ đạo, thực tế của hạt trong không gian:}
\begin{center}
\begin{tikzpicture}
    % Định nghĩa tâm
    \coordinate (O) at (0,0);

    % Vẽ các đường nét đứt tỏa ra từ tâm (12 đường, mỗi đường cách nhau 30 độ)
    \foreach \angle in {0,30,...,330} {
        \draw[dashed, gray!50] (O) -- (\angle:3cm); % Đường dài 3cm, màu xám nhạt, nét đứt
    }

    % Vẽ đường thẳng đứng màu xanh đậm đi qua tâm
    \draw[blue, thick] (0.5, -3) -- (0.5, 3); % Đường thẳng đứng, dày, màu xanh

    % Vẽ điểm tại tâm (một ngôi sao nhỏ)
    \fill (O) circle (2pt); % Điểm tròn tại tâm, bán kính 2pt

\end{tikzpicture}
\end{center}

\end{document}
